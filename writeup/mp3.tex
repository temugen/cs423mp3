\documentclass{article}

\begin{document}

\title{CS423: MP3{}}
\author{Brad Misik and Ivan Hu}

\maketitle

\begin{abstract}
Implementation, development, and case studies of a page fault instrumenting Linux kernel module.
\end{abstract}

\section{Introduction}
Although we read through and implemented the MP3 spec step-by-step, there were quite a few places where our implementation deviated. Those deviations, as well as a brief overview of the project and the case studies involved, will be outlined in this document.

\subsection{Environment Setup}
As with MP2, I used my local virtual machine, because the remote machines were painfully slow to work on. I created a git repo of the root folder of MP2 and uploaded it to share with the team.

\subsection{Initial Development}
The majority of MP3 was just refactoring the solution from MP2. The struct and variable names were modified, some functions were deleted, but the majority of both the structure, variables in use, and functions remained in the code.

\section{Implementation Deviations}

\subsection{Workqueue}
For the work queue, we decided to create it when loading our module and destroy it when exiting our module. We had initially followed the spec and had the queue only exist when tasks were in our task list, but the overhead of keeping it in memory is presumed to be negligible, and it makes intuitive sense to have its life span over the life of our module.

\subsection{Buffer Allocation and Mapping}
In the 2.6 kernel and above, it appears as though the PG\_reserved page flags are being phased out, and the more appropriate (and WAY easier way) of allocation a buffer for use in userspace is the vmalloc\_user() function. Once using this function for buffer allocation, said buffer may be mmaped into a vma by way of the remap\_vmalloc\_range() function. That's it - two simple calls and the blunt of this MP was taken care of.

\subsection{Work and Monitor}
We used a static delayed work structure so we didn't have to go and reinitialize our work every period, or have to worry about timers. This was the most simple solution. Our work handler uses a pointer into our vmalloc'd ring buffer to a sample struct. This sample struct has members for all of the necessary values to be stored each period. We modified the monitor process to use this sample struct to easily iterate through the buffer dumping out its data. The included sample header file also defines the number of (max) samples, to synchronize the buffer information between kernel module and monitor.

\section{Case Studies}

\subsection{Case Study 1: Random vs Local}
Graph 1 had a steeper curve than graph 2, though some of that may be distortion because of the scaling of the x axis. The scaling of the x axis is different because graph 1 had a much longer running time (80189 vs 51502 jiffies), which is because graph 1 was made to test random based access, and graph 2 locality based access. Locality based access is faster since it is likely that adjacent memory values are also in already loaded page tables. The random based access needs to swap out more page tables, especially when its buffer size is larger than physical memory. Graph 1 does appear to have more page faults per unit time, which is easy to answer at the surface, for the same reason that random access causes more page faults than locality based access. However, that also means that it takes more time and therefore needs to take longer to recover from those page faults before it has more, probably leading to a cap on the number of page faults possible in one period. The moral of the story is that locality based access is desired if you want to minimize overall page counts and shorten the time taken for a process.

\subsection{Case Study 2: Multiprogramming}

\subsection{Case Study 3: RT Synchronization}

\end{document}